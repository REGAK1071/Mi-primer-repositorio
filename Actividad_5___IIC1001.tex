\documentclass{article}
\usepackage{graphicx} % Required for inserting images
\renewcommand{\familydefault}{\sfdefault}

\title{\textbf{Actividad 5 - IIC1001}}

\author{Diego Jadue}
\begin{document}

\maketitle

\section{Algoritmo de Euclides}
\paragraph{Es uno de los algoritmos mas antiguos que existen en el mundo de las matemáticas, fue creado por el matemático griego Euclides alrededor del año 300 a.C. }

\subsection{¿Para que sirve?}
\paragraph{El algoritmo de Euclides es un método para encontrar el máximo común divisor (MCD) de dos números enteros positivos mediante un bucle de divisiones hasta encontrar el número más grande que los divida a ambos sin dejar residuo.}

\paragraph{Dentro de sus aplicaciones practicas se destacan las siguientes:}
\begin{itemize}
    \item \textbf{Criptografía y seguridad}: Se utiliza en la generación de claves para el cifrado RSA, un método aún utilizado para asegurar la comunicación digital.
    
    \item \textbf{Simplificación de fracciones}: Es probablemente el uso más común que tenga este algoritmo. Pues se calcula el MCD del numerador y denominador, para luego simplificar la fracción por el MCD.
    
    \item \textbf{Aplicaciones en programación y algoritmos}: En la programación, esta herramienta es muy útil para optimizar y resolver problemas relacionados con divisibilidad y factores comunes.
\end{itemize}

\subsection{Información obtenida de:}

\textrm{
Gomila, J. G. (2024, 23 septiembre). Algoritmo de Euclides, definición, funcionamiento, aplicaciones y ejemplos. 
Frogames. \newline
https://cursos.frogamesformacion.com/pages/blog/algoritmo-de-euclides}

\textrm{ Khan Academy. (s. f.). https://es.khanacademy.org/computing/computer-science/cryptography/modarithmetic/a/the-euclidean-algorithm}

\end{document}